Getting started with FullFAT is easy, in all releases, some demo projects are included. These provide a well-commented reference to the FullFAT API. These demo's are based on my FFTerm project, (FullFAT Terminal) which was spawned through the need of a free and open-source portable and simple terminal program.
\newline
\newline
Demo projects exist for the following platforms:
\begin{itemize}
\item Windows using Visual Studio 2008.
\item Windows using Makefile and the GCC Compiler.
\item Linux or Unix using Makefile and the GCC Compiler.
\end{itemize}

By using these demo's you should be able to write similar code for your platform, and therefore integrate FullFAT very quickly without having to refer to this manual all the time. This manual therefore serves as a reference to the complete FullFAT API, lists any issues with the current release, and provides information about how to gain the best performance from the FullFAT software.

\subsection{Windows using Visual Studio 2008}
	It is very easy to build any of the FullFAT demo projects. For the Visual Studio project, simply open the solution file FullFAT.sln and then click Build->Build Solution. Once built, press the play button to begin debugging. This should open up a FullFAT Terminal window allowing you to use the usual commands.

\subsection{Compiling the Makefile Demos}
To compile FullFAT for these projects you need:
\begin{itemize}
\item GCC (or appropriate C compiler).
\item GNU Make.
\end{itemize}
You can simply type the following on the command-line:
\newline
\newline
make
