The FullFAT API tries to be concise and easy to use, while providing a lot of flexibility to cover the needs of all possible scenarios.

If FullFAT doesn't provide a service or function natively, then it should be very easy for a developer to easily recreate such functionality using the other provided API functions.
\newline
\newline
\textbf{NOTE:} Current API is subject to change before the 1.0 release. Pay particular attention to the FF\textunderscore Open() function, which is likely to match the ANSI C fopen() standard library function for the 1.0 release.


\subsection{FF\textunderscore CreateIOMAN()}

FF\textunderscore CreateIOMAN() creates the FF\textunderscore IOMAN type object, that is required for all FullFAT API functions except the FILE I/O functions. This allows FullFAT to be completely re-entrant throughout any execution context, and also provides the possibility to have multiple concurrent instances of FullFAT running in a single or multi-threaded environment. Inn other words, FullFAT can use multiple devices and partitions, while keeping each device completely isolated.

\begin{lstlisting}
#include "fullfat.h" 				// Include FullFAT's type definitions.

FF_IOMAN *pIoman;					// Declare an I/O Manager pointer.

pIoman = FF_CreateIOMAN();

\end{lstlisting}


\subsection{FF\textunderscore DestroyIOMAN()}

This function deestroys the IOMAN object and cleans all memory or resources used.

This function will only succeed when there are no actively mounted partitions. \textbf{See: FF\textunderscore UnmountPartition}.
